
\documentclass[10pt]{article}

\input{keytemp/config.tex}

\usepackage{graphicx}

\usepackage{fancyvrb}
\usepackage{multicol}
\usepackage{listings}



\lstset{language=bash,
       basicstyle=\footnotesize\ttfamily\scriptsize,
       captionpos=b,
       tabsize=2,
  }

\usepackage{geometry}

\geometry{letterpaper, margin=10mm}

\setlength{\columnsep}{10mm}
\setlength{\columnseprule}{0.3pt}

\setlength{\parskip}{1.2ex}

\setlength{\parindent}{0em}

\pagenumbering{gobble}

\begin{document}

\section*{PaperKey Backup of GPG Key \keyid}

This is a backup of the \keytype \space \keyid, useful for \keyusage. These are made using the paperkey utility, both in human-readable text, as well as a QR code. 
\ifx\keyid\primarykey
Any subkeys of this primary key are too large to be included with it, so they are exported separately, one subkey per page.
\else
This subkey is a part of primary key \primarykey, which was exported separately.
\fi

These formats do not contain the public key information, which will have to be obtained from a public key server when recombining using paperkey.

\begin{multicols}{2}

\subsection*{Fingerprint}

This key can be looked up using the 40-digit fingerprint of \scriptsize 0x\keyfingerprint \normalsize, or with this QR code:

\begin{center}
\includegraphics[width=0.5\columnwidth]{keytemp/\keyid-qrfpr.png}
\end{center}

\subsection*{QR Codes}


Below is the QR code backup. It contains the same hex as a human-readable paperkey, but without hex checksum or formatting. They can be read back in using:
\begin{lstlisting}
$ cat read-from-qr.txt | xxd -r -p | 
paperkey --pubring ~/.gnupg/pubring.gpg -o recover.gpg
\end{lstlisting}

\begin{center}
\includegraphics[width={0.8\columnwidth}]{keytemp/\keyid-qr.png}
\end{center}

\columnbreak

\subsection*{Paperkey}

\fvset{fontsize=\tiny}
% \fvset{fontsize=5.5}
\VerbatimInput{keytemp/\keyid-paperkey.txt}


\end{multicols}
\end{document}
